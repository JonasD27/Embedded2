In dieser Aufgabe sollte das LCD Display in Betrieb genommen werden. Es wurde dafür die Bibliothek \mbox{\hyperlink{lcd__gpio_8h}{lcd\+\_\+gpio.\+h}} geschrieben. Der Zugriff erfolgt dabei noch über die GPIOs. Um das LCD anzusteuern muss dieses vorher einmalig initialisiert werden. Dies geschieht in der Funktion \mbox{\hyperlink{lcd__gpio_8c_ac23e73124dc9fabae95671fe71d074a6}{lcd\+\_\+init()}}.

Ist das LCD initialisiert, können über die Funktion \mbox{\hyperlink{lcd__gpio_8c_a9958d5441d79af42a23a03f6229de09d}{lcd\+\_\+write\+\_\+data()}} Daten über den Bus versendet werden. Mit der Funktion \mbox{\hyperlink{lcd__gpio_8c_a6bcfd73d519832504e3b8bc7b3f6f408}{lcd\+\_\+get\+\_\+status()}} wird das Busy-\/\+Flag und der Adress-\/\+Zähler abgefragt. Die Funktion \mbox{\hyperlink{lcd__gpio_8c_a01b1a4ea4a64b70bfc2f1569bd57bcdc}{wait\+For\+Busy\+LCD()}} liest das Busy-\/\+Flag und blockiert solange dies gesetzt ist. In den anderen Funktionen, welche einen Schreib-\/\+Zugriff implementieren wird immer zuerst die Funktion \mbox{\hyperlink{lcd__gpio_8c_a01b1a4ea4a64b70bfc2f1569bd57bcdc}{wait\+For\+Busy\+LCD()}} ausgeführt um zu gewährleisten, dass die zu sendenden Daten auch vom LCD interpretiert werden können.

Es wurde eine Funktion \mbox{\hyperlink{lcd__gpio_8c_a151abba6f7ca2f5cd5060fbaa26697e8}{write\+Str\+LCD()}} geschrieben, welche es erlaubt komplette Strings auf dem LCD anzuzeigen.

Weiter Funktionen sind \mbox{\hyperlink{lcd__gpio_8c_ad235a86241458b1e7b8771688bfdaf9a}{lcd\+\_\+clear()}}, welche das LCD zurücksetzt und den Inhalt löscht sowie \mbox{\hyperlink{lcd__gpio_8c_acfc4f2670ecb8c7c07fb015adb54a526}{lcd\+\_\+set\+\_\+pos()}}, welche den Cursor auf die gewünschte Position bewegt.

Um die geforderten Timings einzuhalten wurden zwei unterschiedliche Delay-\/\+Mechanismen verwendet\+: \mbox{\hyperlink{lcd__gpio_8h_ad836aa0703fbad4b9fed4a4582496e03}{\+\_\+\+\_\+delay\+\_\+cycles()}} verzögert um die angegebenen Zyklen. In unserem Fall beträgt FCY 50MHZ also ist ein Taktzyklus 20ns lang. \+\_\+\+\_\+delay\+\_\+us() verzögert um die angegebenen Mikrosekunden.

In der Super\+Loop wird die Funktion \mbox{\hyperlink{main_8c_add65c2bbc4a328fc893abeded917bc66}{display\+\_\+temp\+\_\+load()}} jede ms aufgerufen und gibt in einem Takt von 3s abwechselnd die akutelle Temperatur und Aulastung in der ersten Zeile des LCDs aus. Des Weiteren wird alle 3s die per UART empfangenen Daten in der zweiten Zeile des LCDs ausgegeben. 